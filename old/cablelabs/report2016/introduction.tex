\section{Introduction}\label{sec:intro}

As traffic demands increase due to the rise of
large asymmetric traffic flows such as
 video
streaming,
interconnection arrangements must evolve to meet these new demands. The nature, causes,
and location of Internet congestion has spawned contentious debate over
the past two years. End users have become increasingly invested in this
topic as well, although they have sometimes conflated the issues of Internet
congestion with other concerns about the prioritization of Internet
traffic.

Discussion about interconnection can and should be better informed by
accurate, up-to-date information about where capacity bottlenecks
exist. Unfortunately, until now, 
data about traffic utilization at Internet interconnection points has
been hard to come by, due to confidentiality and business
constraints. This opacity has led users, policymakers, and researchers
to resort to techniques that attempt to isolate congestion using
end-to-end probes~\cite{coates2001network,www-ndt,www-mlab}, which
nonetheless still leave significant uncertainty 
about where congestion may be occurring.  

One of the biggest barriers to furthering this debate is the lack of
clear data on this problem. As the Internet pioneer David Clark recently
said, ``An issue that has come up recently is whether interconnection
links are congested. The parties who connect certainly know what's going
on, but that data is generally not disclosed. The state of those links
matters to a lot of people ... and there have been some
misunderstandings around congestion and interconnection
links''~\cite{clark:nanog66}.   

To help shed light on this important issue, ISPs have provided
unprecedented data around the state of interconnection links. This data
yields some information concerning the utilization of network ports that
face each network's ``peers'' (i.e., the networks that each ISP connects
to directly). This data aims to illuminate the utilization properties of
each network’s externally facing switch ports and ascertain whether each
collection of ports between a given ISP and its respective neighboring
network is uncongested. Although this data cannot, by itself, tell the
complete story about the location of congestion along end-to-end
Internet paths, it can tell us a lot about where congestion is not
occurring.  

Each participating ISP has provided information about its interconnection to
neighboring networks (e.g., ISPs, content providers) in each region, as well as the capacity of each interconnect.
The data that participating ISPs provide account for about 97\% of links
from all participating ISPs in any given month; the only links that are
missing from the dataset are those where the measurement infrastructure
has not yet been deployed.  This information offers sufficient
information to ascertain the capacity of each interconnect between an
ISP and neighboring networks. Given this information, we can compare
this provisioned capacity against traffic statistics for traffic that
traverses each of these network ports and compare the measured
utilization to the provisioned capacity to achieve an estimate of
utilization. The ability to perform this analysis depends on the ability
to collect accurate, utilization measurements. Section~\ref{sec:method}
discusses the collection method.

Ideally, the information we would be able to see the utilization
and capacity for each individual port, for every ISP---in such a
scenario, comparing utilization to provisioned capacity would be
straightforward. Of course, the practical realities are more
complicated: even the {\em existence} of an individual interconnection is
typically considered proprietary, not to mention the business agreement
surrounding that interconnection, as well as the capacity and
utilization of the interconnection. As a result, the data that the ISPs
provide aggregates sampled flow statistics across link groups 
in each region, providing a high-level picture of capacity and
utilization per region and ISP, as well as how this utilization
fluctuates over time. The traffic
flow statistics, based on IPFIX~\cite{rfc7011} and
collected by DeepField Networks~\cite{www-deepfield} represents
utilization information that is collected at the interconnection points,
thus providing a more direct indication of the utilization information
at interconnection points.

The data does have some limitations that make it inappropriate for
answering certain questions about utilization. First, it is sampled,
which makes it difficult to answer certain types of questions about flow
size distributions, characteristics of small flows, and utilization by
application.  Second, to preserve proprietary information, the data is
aggregated and anonymized, preventing conclusions about utilization at
specific interconnection points. Yet, the data illuminates
interconnection capacity and utilization at many levels. Throughout this
report, we are careful to highlight conclusions that we can and cannot
make with the data that the participating ISPs have
provided. Based on feedback from other experts, we have also iterated on
the data that the ISPs have agreed to release, resulting in a careful
balancing act between preserving proprietart information and revealing
information about utilization at interconnection 
points that can inform ongoing debates. 
Subsequent sections of this
report provide additional detail on the method used to collect and
report this data, as well as what we can and cannot conclude from the
data that the ISPs have agreed to provide. 

This paper reflects our current understanding of capacity and
utilization at interconnection points; we recognize that the dialog
surrounding interconnection is ongoing.  As a resource to interested
parties---and to promote further academic research in this field, 
we will periodically update the 
findings and data from this project on the project
website~\cite{www-citp-interconnection}. In cooperation with the
participating ISPs, we will annually assess whether the project remains
relevant as Internet interconnection evolves. We also expect
potential future opportunities to correlate this data with
performance measurements from other sources, which will shed more
light into the relationship between utilization at 
interconnection and end-to-end performance.

The rest of this paper is organized as follows.
Section~\ref{sec:related} describes related work and analysis
techniques. Section~\ref{sec:method} describes the measurement techniques and
data, as well as the effects of
various phenomena such as sampling on the accuracy of the collected
data. Section~\ref{sec:applicability} discusses where the measurements
  from this study can (and cannot) be applied. 
  Section~\ref{sec:findings} describes the findings from a preliminary
  analysis of the data collected as part of the
  project. Section~\ref{sec:conclusion} concludes with a summary and
  suggestions for possible next steps.
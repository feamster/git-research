\section{Conclusion and Next Steps}\label{sec:conclusion}

Public discourse surrounding interconnection and congestion begs the need
for better visibility into congestion at interconnection points between
ISPs and content providers.  Unfortunately, the methods that exist for inferring
these statistics from the edge using active probes are
inconclusive---cannot accurately pinpoint congestion at interconnection,
and in many cases they cannot disambiguate congestion that occurs on a
forward path from congestion that occurs on a reverse path.

Ultimately, stronger conclusions require more direct measurements of
utilization at the interconnection points themselves. The public data
collected from ISP interconnection points makes it possible to establish
that spare capacity exists at interconnection points in the aggregate,
and that congestion that is observable at the edge may ultimately
reflect the inefficient use of existing capacity.  Until now, all of
this information has been protected by non-disclosure agreements
between ISPs and neighboring networks. Yet, more informed debate requires better data.  This
paper presents a next step in that direction, based on data from
interconnection points from seven major Internet service providers. 

Our preliminary analysis tells a different story than previous direct
measurement approaches have suggested. Specifically, evidence suggests
that, capacity continues to be provisioned to meet growing
demand and that spare capacity does exist at interconnection points,
even though specific links may be experiencing high utilization. We do not speculate on the reasons behind these usage
patterns, which ultimately derive from content (``edge'') providers'
decisions about where to direct traffic, but the patterns appear to show
clear trends: there exists spare capacity at the interconnection points.

The need to assess metrics that directly affect user experience, such as
application quality or the quality of user experience, 
will ultimately require a much richer dataset than that which is
currently available. For example, more work is needed to understand how
the utilization of a link ultimately affects a customer's quality of
experience for a given application. It may be possible, for example,
that high utilization does not adversely affect customer quality of
experience. Future work may include assessing the correlation between
these network-level traffic statistics and the corresponding quality of
experience for different types of applications. 

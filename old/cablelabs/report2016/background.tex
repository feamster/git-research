\section{Background: Interconnection}\label{sec:background}

Interconnection often involves complex business relationships between
the interconnecting parties. The simplest taxonomy relates to the nature
of the interconnection itself, and who pays for the traffic that
traverses the interconnect (and, as a result, who is responsible for
determining the capacity of that interconnect). 
\begin{itemize}
\itemsep=-1pt
\item In a transit
relationship, one network (the ``customer'') pays the other (the
``provider''), regardless of the direction in which traffic flows. In a
settlement-free interconnection relationship, parties exchange traffic
without payment (``settlement'').  In this paper, we refer to these
types of links as {\em ISP-sized} interconnections.
\item In a settlement-free
interconnection relationship, the parties exchange traffic between each
other’s customer networks; specifically, they do not transit traffic to
other provider networks or settlement-free interconnect partners.  In
this paper, we refer to these types of links as {\em jointly-sized}
links because capacity decisions in these relationships are typically
made in collaboration with another ISP.
\item A paid
peering relationship is similar to a settlement-free interconnection in
that pairs of ISPs will exchange traffic of each other’s customers; the
difference is that in a paid peering relationship, one ISP will
typically pay the other for the interconnection arrangement.  In these
relationships, capacity decisions are, by contract, often made by
another entity than the ISP---typically a large content or ``edge''
provider that has a significant amount of traffic to send. In this
paper, we refer to these types of links as {\em partner-sized}
interconnections. 
\end{itemize}
\noindent
In practice, these interconnection relationships may be complicated by
other relationships and arrangements that affect performance and cost,
other than the interconnect itself. For example, a business relationship
may involve arrangements for the placement of caches with various
capacity; one ISP may sometimes agree to place caching servers in lieu
of a settlement-free interconnection or paid peering relationship. 

We find that the interconnection relationships
that ISPs have with one another can in some cases have bearing on the
utilization of the respective interconnection links between the
ISPs. For example, based on our initial analysis of five months worth of
interconnection data, we can conclude that paid peering relationships
tend to operate at significantly higher utilization than interconnection
links that support other types of business relationships (e.g.,
transit).


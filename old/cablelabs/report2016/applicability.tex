\section{Limitations}\label{sec:applicability} 

In this section, we briefly discuss the applicability of the measurement
techniques for various purposes. We survey the types of conclusions
can and cannot be drawn from sampled and aggregated IPFIX measurements. 

\subsection{Limitations of Flow Statistics}

Traffic flow statistics are commonly used to estimate link utilization
for purposes of capacity estimation and planning, and for traffic
engineering purposes. Large transit provider networks commonly deploy
IPFIX across all of the routers in their networks to determine whether
certain links are overutilized.  As previously discussed, even sampled
IPFIX records can be useful for determining {\em aggregate} link
utilization.  Nonetheless, sampled IPFIX records have certain
limitations that make them inappropriate for certain types of
analysis. While these additional features would undoubtedly shed more
light on both congestion and application performance, the currently
deployed technologies do not permit these types of analyses at the
interconnection points.  The rest of the section discusses various
measurements that are not possible with the existing measurement
approach.

\paragraph{Analysis of small flows.} Due to the sampling rates of the
measurements, performing any analysis that is specific to small flows or
on the distribution of flows may not be possible. As previously
discussed, this affects our ability to analyze statistics such as
flow-size distribution but should not have any affect on our ability to
estimate utilization.

\paragraph{Timing, loss, or quality of experience.} Traffic flow
statistics also do not capture timing effects or accurate statistics
about packet loss, jitter, and so forth. Due to the lack of detailed
information that aggregate traffic flow statistics provide, inferring
properties that directly relate to user quality of experience will be
difficult with the existing dataset, given only aggregate volumes. 

\paragraph{Information about specific applications.} Additionally,
assessing the performance of any given application will be difficult
with the given dataset, since the traffic flow statistics do not have
any application-specific identifiers or other information that would
help associate the traffic with a particular end-user
application. Traffic flow statistics are gathered on
flows, which correspond to source and destination IP address and port,
as well as protocol type. Yet, this information alone does not provide
enough information to infer the application type of a flow, since
applications often share the same destination port (in particular, many
applications, including streaming video and the web, use destination
port 80). Associating performance with a particular application
will require more precise statistics, including possibly information from the
application layer or associated domain name system (DNS) lookup
information.

\paragraph{Statistics on short timescales.} The
traffic statistics represent aggregates across a group of links and
across time (typically the duration of a particular flow). As a result,
the statistics cannot capture fluctuations that may occur on short
timescales; for example, a traffic flow may send a high volume of
traffic over a relatively short interval and low volume for the
remainder of the flow duration. 
Utilization may spike on short timescales, and such spikes
would not be reflected in aggregate traffic flow statistics, since one
can really only compute an average utilization over the duration of time
that the flow record reflects. 
Because the aggregate statistics reflect
only average utilization across the duration of a flow, the statistics
will reflect these short-term fluctuations. 

\subsection{Limitations Due to Aggregation}

Even in the private dataset, statistics are reported in aggregate link
groups. In this case, any fluctuations that occur on only a single link
may not be reflected in the aggregate statistics. We previously
described assumptions about traffic load balance that suggest that
drawing conclusions based on average utilization per link is reasonable.
Additionally, short-term periods of high utilization across the entire
link group may not be evident in the data, because utilization is
reported on five-minute averages.


In the public dataset, it is possible assess the overall utilization in
some region across all ISPs and partner networks, but not for any
individual interconnection point in a region.  Similarly, it is possible
to see the aggregate utilization for any of the participating ISPs, but
not for a specific region or neighboring AS.  As a result, the aggregates
make it difficult to drill down into the utilization between any pair of
networks, either as a whole or for any particular region.  As a result,
it is not possible to conclude that no interconnection links experience
high utilization. Because the public data shows utilization across each
ISP, we can conclude that each ISP has spare capacity---although we
cannot conclude that it has spare capacity in each region or on any
individual port.  

To mitigate concerns that result from this level of aggregation, the
public dataset also includes 95th percentile peak utilizations for all
links in the dataset, which demonstrates that most of the links in the
dataset as a whole experience low utilization, and that much of the
aggregate capacity remains under-utilized even at peak. We also show the
aggregate utilization for all ISPs in each region, which allows us to
demonstrate that each region has spare capacity; because this statistic
is aggregated across ISP, we cannot conclude that a particular ISP has
spare capacity in a region---especially to a specific neighbor.  Yet,
the our ability to show spare capacity in aggregate increase confidence
that this capacity exists, since most ISPs have significant spare
capacity at peak utilization, and most links in the dataset have spare
capacity at peak, as well.


% Although the public data does not make it possible to conclude that a
% particular ISP is uncongested, it should be possible to determine that
% there exist uncongested ports in any link group. For example, even if
% there exists some subset of congested links in a link group, the average
% utilization across the aggregate would reflect an average aggregate link
% group utilization that is less than the overall capacity of the link
% group. 

\begin{abstract}
The rapidly evolving nature of interconnection has sparked an increased interest
in developing methods for gathering and collecting data about 
utilization at interconnection points. One mechanism, developed by DeepField Networks, allows
Internet service providers (ISPs) to gather and aggregate utilization
information using network flow statistics, standardized in the Internet
Engineering Task Force as IPFIX. This report (1)~provides an overview of
the method that DeepField Networks is using to measure the utilization of
various interconnection links between content providers and ISPs or
links over which traffic between content and ISPs flow; and (2)~surveys
the findings from five months of Internet utilization data provided by
seven participating ISPs---Bright House
Networks, Comcast, Cox, Mediacom, Midco, Suddenlink, and Time Warner
Cable---whose access networks represent about 50\% of
all U.S. broadband subscribers.

We first discuss the problem of interconnection and utilization at
interconnection points. We then discuss the basic operation of the
measurement capabilities, including the collection and aggregation of
traffic flow statistics (i.e., IPFIX records), providing an assessment
of the scenarios where these aggregate measurements can yield accurate
conclusions, as well as caveats associated with their collection.  We
assess the capabilities of flow statistics for measuring utilization,
and we discuss
the capabilities and limitations of the approach
the aggregation techniques that the ISPs use both in providing data to
us, and that we apply before making the data public. 

The dataset includes about 97\% of the paid peering, settlement-free
peering, and ISP-paid transit links of each of the participating ISPs.
Initial analysis of the data---which comprises more than 1,000 link
groups, representing the diverse and substitutable available
routes---suggests that many interconnects have 
significant spare capacity, that this spare capacity exists both across
ISPs in each region and in aggregate for any individual ISP, and that the aggregate utilization across
interconnects is roughly 50\% during peak periods.

% We couple this report with the announcement of an ongoing project at the
% Princeton Center for Information Technology Policy (CITP): The {\em
%   Interconnection Measurement Project}, which will curate information
% about utilization at Internet interconnection points in an effort to
% foster ongoing evidence-based discourse surrounding this important topic.
\end{abstract}